\section{Tools \& Technologies}
\subsection{Languages}
\begin{itemize}
  \item HTML
  \item CSS 
  \item JavaScript 
  \item Python
  \item PostgresSQL

\end{itemize}
\subsection{Next.js}
\subsection{FastAPI}
\subsection{Postgres(Supabase)}
\subsection{Scikit-Learn}

\section*{Justification for Choice of Technology Stack}

The technology stack for the medical prediction website was carefully chosen to balance performance, scalability, and ease of development. Next.js was selected for the frontend because it provides a modern React framework with support for server-side rendering and static site generation, ensuring fast page loads and a responsive user interface. FastAPI was chosen for the backend due to its lightweight design, asynchronous capabilities, and automatic API documentation, which allows for efficient handling of requests and easy integration with machine learning models. PostgreSQL, accessed via Supabase, serves as the database system because it is reliable, scalable, and supports complex queries while Supabase provides an easy-to-use backend-as-a-service interface that simplifies authentication and data management. Finally, Scikit-Learn was used for the machine learning component due to its simplicity, extensive functionality for classification tasks, and seamless integration with Python, enabling quick development and deployment of the pretrained HCV prediction model. This combination of technologies ensures that the system is both performant and maintainable while allowing for future scalability and enhancements.
