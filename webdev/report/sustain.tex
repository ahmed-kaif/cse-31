\section{Sustainability \& Environmental Considerations}

In developing the medical prediction website, sustainability and environmental 
impact have been considered in both design and deployment. The following strategies are applied:

\subsection*{Performance Optimization}
\addcontentsline{toc}{subsection}{Performance Optimization}
The backend, implemented using FastAPI, is lightweight and asynchronous,
which reduces server load and ensures efficient handling of requests.
The frontend, built with Next.js, supports static site generation and
server-side rendering, which decreases repeated computation and improves
response times. These optimizations reduce unnecessary resource consumption
and improve the overall energy efficiency of the system.

\subsection*{Energy Efficiency in Coding \& Infrastructure}
\addcontentsline{toc}{subsection}{Energy Efficiency}
The codebase follows modular and efficient practices to minimize redundant operations.
Database queries are optimized to avoid excessive computation in SQLite.
Containerization with Docker allows for consistent and isolated environments,
reducing wasted resources across deployments. Additionally, horizontal scaling strategies
are preferred over overprovisioning, ensuring resources are used only when required.

\subsection*{Long-Term Impact}
\addcontentsline{toc}{subsection}{Long-Term Impact}
By deploying the application on modern cloud platforms such as Render\footnote{\url{https://hcv-ai.onrender.com/}}
the system benefits from shared, energy-efficient infrastructure.
This reduces the need for dedicated high-power servers.
The chosen stack (FastAPI+Next.js) supports scalability, which means the application
can grow without significant increases in energy use per user.
In the long term, the platform aims to provide medical predictions with minimal environmental
footprint while supporting sustainable digital health practices.

